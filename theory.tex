\chapter{Introduction}
\label{mathchapter}

\section{Introduction}\label{sec:intro}

At the forefront of the interconnection between particle physics and cosmology are the following questions: (1) What is the origin of the matter-antimatter 
asymmetry?; (2) What is the origin of neutrino mass?; (3) Are there new fundamental forces in nature?; (4) What is the origin of dark energy; and (5) Is the Higgs 
boson solely responsible for electroweak symmetry breaking and the origin of mass? Much like the Higgs mechanism is introduced to account for the SU(2)xU(1) 
symmetry breaking, there are a plethora of theoretical models which incorporate additional gauge fields and interactions to address these questions. For example, 
string theory is considered a promising candidate for describing gravitational systems at strong coupling and thus plays a prominent role in the description of 
black holes and evolution of the universe through the understanding of the origin of dark energy. Similarly, models with additional neutrino fields at the TeV 
scale provide a possible explanation for the mass of light neutrinos. Such models often manifest themselves as new heavy particles that could be observed at the 
LHC. Surprisingly, some of these new particles predicted on the basis of pure particle physics arguments can even provide the correct dark matter relic density. 
There are several ways new heavy gauge bosons appear. The most natural possibility is one in which these heavy gauge bosons are the gauge field of a new 
local broken symmetry. Examples include models with a new U(1) gauge symmetry, little Higgs models, and E6 Grand Unified Theories (GUT). 
In models with a new U(1) gauge symmetry, the $Z^\prime$ is the gauge boson of the broken symmetry. In
little Higgs models, breaking of the global symmetry by gauge and Yukawa interactions generates Higgs mass and couplings at the TeV scale that cancel off the SM
quadratic divergence of the Higgs mass from top, gauge, and Higgs loops. This results in one or more $Z^\prime$ bosons. In Kaluza-Klein models, the $Z^\prime$
bosons are excited states of a neutral, bulk gauge symmetry.
From the breadth, scope, and implications of these models, it is apparent that probing these questions and puzzles potentially lies in the physics of new
particles at the TeV scale. 

Of particular interest for this analysis are models that include an extra neutral gauge boson that decays to pairs of high-$p_{T}$ $\tau$ leptons. Although 
many models with extra gauge bosons obey the universality of the couplings, some models include generational dependent couplings resulting in extra neutral gauge 
bosons that preferentially decay to $\tau$ leptons, making this analysis an important mode for discovery. However, even if a new gauge boson decaying to $\mu\mu$ 
is discovered first, it will be critical to establish the $\tau\tau$ decay channel to establish the coupling relative to $\mu\mu$ channel.
