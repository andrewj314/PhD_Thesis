\chapter{Introduction}

\section{Introduction}\label{sec:intro}

The mission of particle physics is to increase our understanding of the most fundamental constituents of our universe and their interactions. These efforts have arguably been going on since the era of the Greek philosophers, who classified all matter into the elemental categories of earth, air, wind, and fire. As the centuries progressed, our knowledge of the nature of the universe has been refined as new theories were proposed to explain the phenomena we observe in nature, and, at the same time, more and more sophisticated experiments were designed to test them. Particle physics as a field today is broadly divided into two sub-fields: theory and experiment. Theorists seek to develop models that offer a more complete explanation of particle interactions, and experimentalists are tasked with validating these models. Efforts in particle theory in the late-20th century culminated in the Standard Model (SM) of particle physics, which has proven to be tremendously successful at describing a large number of observed particle phenomena (some which have been accurately validated to one part in ten billion!).

The SM is not without its deficiencies, however. Today, there remain many open questions in particle physics that are not sufficiently (or at all) addressed by the SM. To name just a few: 

Why is there more matter than antimatter in the universe?
What is dark matter?
What is dark energy?
Can the strong iteraction be unified with the electroweak interaction?
Why is the Higgs mass 125 GeV and not at the Planck scale?

Attempts to answer these questions require new theories modifying and building on the SM. These theories must then be tested, which is where the experimentalists come in. While the SM was perhaps the crowning achievement of high energy theory in the twentieth century, the Large Hadron Collider (LHC) is arguably the most noteworthy undertaking in high energy experiment in the twenty-first. Located in Geneva, Switzerland at the European Organization for Nuclear Research (CERN, from the French ``Conseil Européen pour la Recherche Nucléaire"), the LHC is the highest-energy particle collider ever built and has been an invaluable tool in the quest to validate theories of new physics.

This thesis presents an effort to use one of the primary experiments on the LHC, the Compact Muon Solenoid (CMS), to search for a new particle, called the \emph{$Z\prime$}, predicted by many such ``Beyond Standard Model" (BSM) theories. Part I will lay the groundwork, offering an overview of the SM and the modifying theories which predict the $Z\prime$'s existence. Part II will discuss the tools used in the search: the LHC, the CMS experiment, and the substantial computing resources needed to conduct a full search for new physics. Part III will discuss the search during the 2012 data-taking run at the LHC, and Part IV will discuss the most recent search conducted during the 2015 run. Part V will offer an overview of future possibilities in this search effort as well as concluding remarks.

\part{\huge Part I}

\chapter{The Standard Model}

At the forefront of the interconnection between particle physics and cosmology are the following questions: (1) What is the origin of the matter-antimatter 
asymmetry?; (2) What is the origin of neutrino mass?; (3) Are there new fundamental forces in nature?; (4) What is the origin of dark energy; and (5) Is the Higgs 
boson solely responsible for electroweak symmetry breaking and the origin of mass? Much like the Higgs mechanism is introduced to account for the SU(2)xU(1) 
symmetry breaking, there are a plethora of theoretical models which incorporate additional gauge fields and interactions to address these questions. For example, 
string theory is considered a promising candidate for describing gravitational systems at strong coupling and thus plays a prominent role in the description of 
black holes and evolution of the universe through the understanding of the origin of dark energy. Similarly, models with additional neutrino fields at the TeV 
scale provide a possible explanation for the mass of light neutrinos. Such models often manifest themselves as new heavy particles that could be observed at the 
LHC. Surprisingly, some of these new particles predicted on the basis of pure particle physics arguments can even provide the correct dark matter relic density. 
There are several ways new heavy gauge bosons appear. The most natural possibility is one in which these heavy gauge bosons are the gauge field of a new 
local broken symmetry. Examples include models with a new U(1) gauge symmetry, little Higgs models, and E6 Grand Unified Theories (GUT). 
In models with a new U(1) gauge symmetry, the $Z^\prime$ is the gauge boson of the broken symmetry. In
little Higgs models, breaking of the global symmetry by gauge and Yukawa interactions generates Higgs mass and couplings at the TeV scale that cancel off the SM
quadratic divergence of the Higgs mass from top, gauge, and Higgs loops. This results in one or more $Z^\prime$ bosons. In Kaluza-Klein models, the $Z^\prime$
bosons are excited states of a neutral, bulk gauge symmetry.
From the breadth, scope, and implications of these models, it is apparent that probing these questions and puzzles potentially lies in the physics of new
particles at the TeV scale. 

Of particular interest for this analysis are models that include an extra neutral gauge boson that decays to pairs of high-$p_{T}$ $\tau$ leptons. Although 
many models with extra gauge bosons obey the universality of the couplings, some models include generational dependent couplings resulting in extra neutral gauge 
bosons that preferentially decay to $\tau$ leptons, making this analysis an important mode for discovery. However, even if a new gauge boson decaying to $\mu\mu$ 
is discovered first, it will be critical to establish the $\tau\tau$ decay channel to establish the coupling relative to $\mu\mu$ channel.
