\subsection{Muon Reconstruction and Identification}\label{ss:muonreco}

Muon reconstruction is a multistep process that begins with the information gathered from the muon subdetector. As a
first step, standalone muons are reconstructed from hits in the individual drift tube (DT) and
cathode strip (CSC) chambers. Hits from the innermost muon stations are combined with hits in the other muon segments
using the Kalman fitting technique. The standalone muon trajectory is reconstructed by extrapolating from the
innermost muon station to the outer tracker surface. This standalone trajectory is then used to find a matching track
reconstructed in the inner silicon tracker. Finally, standalone muons and matching silicon tracks are used to perform a
global fit resulting in a ``global'' muon. Muon reconstruction is described in more detail in \cite{CMS_MUO_10-004}.

Global muons are reconstructed by combining tracker muons from the inner silicon tracker and standalone muons from the
muon chambers. Once a muon is required to have matching tracks in the inner and outer detectors,
the main source of background consists of charged hadrons that leave a
signature in the inner silicon tracker while also penetrating through the hadronic calorimeter and creating hits in the
muon chambers. Charged hadrons that penetrate the hadronic calorimeter and leave hits in the muon system will deposit 
significant energy in the calorimeters. Therefore, a calorimeter compatibility algorithm can be used to significantly 
reduce the number of charged pion fakes. However, calorimeter compatibility is not used in this analysis due to our 
uncertainty of the performance of such algorithms in the presence of high PU. 
%Furthermore, as muons are only used to 
%validate tau identification and the trigger performance, the use of the calorimeter compatibility is not required. 
The presence of punch-throughs often occur due to pions from the fragmentation of quarks and gluons. These punch-throughs 
can often be discriminated against by requiring isolation. Similarly, non-prompt muons from heavy flavor decays and 
decays in flights are expected to be within jets and can be discriminated against by imposing an isolation requirement. 
Muon identification is described in more detail in \cite{CMS_MUO_10-004} and \cite{CMS_MUO_11-001}.

Isolated muons are required to have minimal energy from PF neutral and charged candidates in a cone of $\Delta R =
0.4$ around the lepton trajectory. PF charged candidates considered in the calculation of isolation are required to be near the 
primary vertex. Isolation for muons is defined as:

\begin{equation}
   I = \frac{\sum_{i} p_{T}^{i}}{p_{T}^{\mu}}
\label{eq:muIso}
\end{equation}

where the index $i$ runs over PF neutral and charged candidates. Table
~\ref{table:muonidcuts} shows the complete list of for the ``isMedium" $\mu$ identification criteria used in this analysis.
In all channels, the identification and isolation used follows the POG recommended criteria. 
%The exact discriminator names and working points for each channel are listed and described in their respective sections.
%The muon trigger/identification efficiencies and scale factors used to correct the MC expectations in these analyses have been taken from: 
%``https://twiki.cern.ch/twiki\\/bin/viewauth/CMS/MuonReferenceEffs\#22Jan2013\_ReReco\_of\_2012\_data\_re"

\begin{table}[ht]
  \caption{$\mu$ Identification}
  \centering{
  \begin{tabular}{| l | c |}
  \hline\hline
        Cut \\ [0.5ex] \hline
        $\textrm{muon::isLooseMuon(recoMu)}$ \\
        $\textrm{recoMu.innerTrack()-}$$>\textrm{validFraction()}$$> 0.8$ \\
        AND \\
        $\textrm{recoMu.globalTrack()-}$$>\textrm{normalizedChi2()}$$< 3$ \\
        $\textrm{recoMu.combinedQuality().chi2LocalPosition}$$< 12$ \\
        $\textrm{recoMu.combinedQuality().trkKink}$$< 20$ \\
        $\textrm{muon::segmentCompatibility(recoMu)}$$> 0.303$ \\
        OR \\
        $\textrm{muon::segmentCompatibility(recoMu)}$$> 0.451$ \\
        RelIso $< 0.15$ \\
  \hline
  \hline
  \end{tabular}
  }
  \label{table:muonidcuts} % is used to refer this table in the text
\end{table}
