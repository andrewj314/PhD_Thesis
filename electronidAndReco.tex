\section{Electron Reconstruction and Identification}


\label{eID}

Electrons are reconstructed using information from the  tracker and Ecal detectors. Electrons
passing through the  silicon tracker material lose energy due to Bremsstrahlung  radiation. The
energy of the radiated photons is spread  over several crystals of the Ecal detector along the 
electron trajectory, mostly in the $\phi$ direction  (the magnetic field is in the z direction).
Two algorithms  based on energy clustering, ``Hybrid'' for the barrel and  ``Island'' for the
endcaps, are used to measure the energy  of electrons and photons \cite{eRecoCMS}.

Electron tracks are reconstructed by matching trajectories  in the silicon strip tracker to seed
hits in the pixel  detector. A pixel seed is composed of two pixel hits  compatible with the beam
spot. A Gaussian Sum Filter (GSF)  is used for the reconstruction of trajectories in the  silicon
strips. In order to minimize the many possible  trajectories due to different combinations of hits,
the  track that best matches an energy supercluster in the Ecal  is chosen to be the reconstructed
track.

The preselection of primary electron candidates requires  good geometrical matching and good
agreement between the  momentum of the track and the energy of the ECAL  supercluster. Two
quantities used to estimate the geometrical matching are $\Delta \eta_{in} = \eta_{sc}
-\eta^{Track}_{vertex}$ and  $\Delta \phi_{in} = \phi_{sc} -\phi^{Track}_{vertex}$. The $\eta_{sc}$
and $\phi_{sc}$ coordinates correspond to  the supercluster position and are measured using an
energy weighted algorithm. The $\eta^{Track}_{vertex}$ and  $\phi^{Track}_{vertex}$ coordinates are
the position of the track at the interaction vertex extrapolated,  as a perfect helix, to the Ecal
detector. The good energy-momentum matching is measured by taking the ratio  between the corrected
energy $E_{corr}$ in the Ecal  supercluster and the momentum of the track $P_{in}$  measured in the
inner layers of the tracker.

Electron selections have two main components, electron  identification (eID) and electron isolation. 
In this analysis we use the non-triggering MVA electron identification. 
The MVA cuts used to define the 80\% and 90\% signal efficiency working points are summarized in Table~\ref{eIDtable}. 
%Note that detector based isolation, instead of PF based isolation, is used in the single electron trigger paths we are using in this analysis.
%In addition, electrons which arise from photon conversions are removed  by requiring that the track associated with the electron to have hits in the inner layers 
%of the pixel detector. 
In all channels, the identification and isolation used follows the POG recommended criteria. 
The exact discriminator names and working points for each channel are listed and described in their respective sections.
%The electron trigger/identification efficiencies and scale factors used in these analyses have been taken from:
%``https://twiki.cern.ch/twiki/bin/view/Main/EGammaScaleFactors2012\#2012\_8\_TeV\_Jan22\_Re\_recoed\_data"

\begin{table}[ht]
\begin{center}
 \caption{Electron ID Selections.\label{eIDtable}}
 \begin{tabular}{| l | c | c | c |}
 \hline\hline
       Category                              & MVA$_{\textrm{min cut}}$ (80\% signal eff)	& MVA$_{\textrm{min cut}}$ (90\% signal eff)  \\[0.5ex] \hline
       Barrel ($\eta < 0.8$) $p_{T}$ 5--10           	& 0.287435 & -0.083313             	\\
       Barrel ($\eta > 0.8$) $p_{T}$ 5--10           	& 0.221846 & -0.235222             	\\
       Endcap $p_{T}$ 5--10           			& -0.303263 & -0.67099             	\\
       Barrel ($\eta < 0.8$) $p_{T}>10$            	& 0.967083 & 0.913286             	\\
       Barrel ($\eta > 0.8$) $p_{T}>10$            	& 0.929117 & 0.805013             	\\
       Endcap $p_{T}>10$          		  	& 0.726311 & 0.358969             	\\
 \hline
 \hline
 \end{tabular}
\end{center}
\end{table}
