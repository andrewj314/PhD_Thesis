\chapter{Search for $Z^{\prime}\to\tau\tau$ events at $\sqrt{s} = 8$ TeV and $\sqrt{s} = 13$ TeV}

\section{Strategy}\label{sec:strategy}
The $\tau$ lepton is the heaviest known lepton with a mass of 1.777 GeV and a lifetime of $2.9\times10^{-13}$ seconds. Around one third of
all $\tau$ leptons decay to $e/\mu$, and the remainder decay into hadronic jets ($\tau_{h}$). 
In the latter case, a $\tau_{h}$ corresponds to one, three, or (rarely) five charged mesons usually accompanied by one or
more neutral pions. Each $\tau$ also decays into a $\tau$ neutrino and, in the case of the $e$ and $\mu$ decays, an additional $e$ or $\mu$ antineutrino.

%As the $\tau$ lepton decays to $e\overline{\nu}_e}\nu_{\tau}~(17.8\%)$, $\mu\overline{\nu}_\mu}\nu_{\tau}~(17.4\%)$, and hadrons +
%$\nu_{\tau}~(64.8\%)$ we will refer to these decays as the $e$ decay channel, $\mu$ decay channel, and the  $\tau_\textrmr{h}$ decay
%channel. 
We consider four distinct analyses for pairs of $\tau$ lepton decays, namely \emu ~(6.2\%), \etau ~(23.1\%), \mutau  ~(22.5\%), and
\ditauhad ~(42\%), where the percentages indicate the branching fraction (probability of decay) for each channel. We ignore the \EE and \MM channels due to the copious Drell-Yan
Z/$\gamma^*\rightarrow e^+e^-, \mu^+\mu^-$ production, although there is ongoing development of algorithms to discriminate prompt $e/\mu$ from $\tau$ lepton 
decays to light leptons, which will be utilized in the 2016 data/analysis.

The overall strategy of the analysis is similar in all channels. 
In general, we identify events with two oppositely charged, nearly back--to--back objects. Because the $\tau\tau$ system decays with up to four neutrinos, we expect to have missing transverse energy (\MET) present in the event. In contrast to \zprime ~searches in the $e^+e^-$ and $\mu^+\mu^-$ channels, the visible \ditau ~mass does not produce a narrow peak due to the missing neutrinos. Instead, we look for a broad enhancement in the \ditau$+$\MET ~invariant mass distribution consistent with new physics.  Our selections maintain high efficiency for signal events, provide strong background suppression, and reduce the influence of systematic effects.  

As \zditau ~is both background as well as an important validation signal, our final selection requirements are such that by removing or
reversing just a few cuts we can obtain a clean sample of \zditau ~events. In order to ensure robustness of the analysis and our
confidence in the results, whenever possible we rely on the data itself to understand and validate the efficiency of reconstruction
methods as well as the estimation of the background contributions. For that purpose, we define control regions with most of the
selection criteria similar to what we use in our main search, but enriched with events from background processes. Once a background enhanced
region is created, we measure the selection efficiencies in those regions and  extrapolate to the region  where we expect to observe our
signal. In cases where a complete data--driven method is not possible, we make use of scale factors, the ratio between observed data events
and expected Monte Carlo (MC) events in the background enhanced region to estimate the background  contribution in the signal region. Monte Carlo events are simulated physics events, wherein a specific physical process (e.g. $Z^\prime \to \tau\tau$) is modeled via computer simulation and fed into a simulated CMS detector. MC simulation is extremely useful for particle physics analyses, as it lets analyzers predict the performance of the detector as well as tune their selection cuts. Although each
individual channel could have its own set of requirements, we maintain, wherever possible, consistent definitions and selection
criteria between channels.

To quantify the significance of any possible excess or set upper limits on the production rate, we perform a fit of the $m(\tau_{1},\tau_{2},\MET)$ mass 
distribution and employ the CL$_{s}$ technique to interpret the results in terms of the upper 95\% credibility level limits
for each channel. The joint limit is obtained by combining the posterior probability density functions (likelihood) and taking into
account correlation of systematic uncertainties within and across channels.

As there are substantial similarities between the analyses performed at 8 TeV and at 13 TeV, this chapter will present the more current 13 TeV analysis carried out in 2015. The 8 TeV results will be included in Section~\ref{sec:results}. The structure of this chapter is as follows: Sections~\ref{sec:samples} and~\ref{sec:triggers} describe the data sets used in each analysis.  Section~\ref{sec:recoID} provides a brief discussion of the reconstruction and identification of the objects used to reconstruct our  $\tau\tau$ 
pairs.
%, while Section~\ref{sec:commonSelection} describes the common selection across the four different channels.
Sections~\ref{sec:muTauhad}--\ref{sec:eMu} describe the specific selection criteria and background extraction methods applied  to each
individual channel. In Sections~\ref{sec:systematics}--\ref{sec:results} we describe the statistical method used to extract 
the $95\%$ C.L. upper limits and the results of the analysis. 
%Finally, Section~\ref{sec:conclusions} summarizes the results and 
%overall sensitivity.
  
