\section{Systematic Uncertainty}\label{sec:systematics}

The following systematic effects have been considered (summarized in Table~\ref{table:SystematicsTable}):

\begin{itemize}
%  \item \textbf{Parton Distribution Functions (PDF):} The systematic effect due to imprecise knowledge of the parton 
%distribution functions is determined by comparing CTEQ6.6L, MSTW2008nnlo, and NNPDF20 PDF with the default PDF and 
%variations within the family of parametrization \cite{CTEQ}. The maximal deviation from the central value is used the overall 
%systematic due to PDFs. We obtain a value of 6.5\%.
%
%  \item \textbf{Initial State Radiation (ISR) and Final State Radiation (FSR):} The systematic effect due to imprecise 
%modeling of initial and final state radiation is determined by re-weighting events to account for effects such as 
%missing a terms in the soft-collinear approach \cite{softCollinear} and missing NLO terms in the parton shower approach \cite{partonShower}. We 
%obtain uncertainties of 0.9\% and 1.2\% for ISR and FSR respectively.

  \item \textbf{Luminosity:} We include a 5\% uncertainty on the measured luminosity\cite{REFLUMI}. It is considered 100\% correlated across MC based 
backgrounds within a channel. It is also considered 100\% correlated across channels (for MC-based backgrounds).

  \item \textbf{Trigger, Reconstruction, and Selection:} 
  An overall uncertainty is applied for the trigger uncertainties determined on the 
  correction factors described in Section 3 and which are measured using tag-and-probe methods. 
  %We consider 6.8\% uncertainty per hadronic tau leg \cite{CMS-PAS-TAU-11-001}.
  %Scale factors for $\tau_{h}$ identification are taken from the tau POG and obtained using a fit of data in a Z$\to\tau\tau$ enhanced region and fixing the cross section to that measured using ee/$\mu\mu$. 
  The uncertainty for muons and electrons is estimated to be 1\% each independent of $\eta$ or $p_{T}$, while we consider 5\% uncertainty per hadronic tau leg in 
the $\tau_{h}\tau_{h}$ channel (assuming each leg is 100\% correlated). The trigger uncertainty is considered 100\% correlated across MC based backgrounds within 
a channel. It is also considered 100\% correlated across channels using the same trigger. For the case of the $\tau_{h}\tau_{h}$ trigger, where the trigger 
efficiency uncertainty is measured per $\tau_{h}$ leg, the total trigger uncertainty is calculated by assuming both legs are 100\% correlated. For example, if the 
per leg $\tau_{h}$ trigger uncertainty is 5\%, the total trigger uncertainty for the $\tau_{h}\tau_{h}$ channel will be 10\%.

  \item \textbf{$b$-Tagging Efficiency (b ID):} We consider a 30\% uncertainty on the mis-tag rate as measured by the 
b-tagging POG\cite{CMS_PAS_BTV_11-001}. For the case of our signal, the systematic 
uncertainty on the requirement of 0 jets mis-tagged as b-jets is determined by propagating the 30\% uncertainty on the 
mis-tag rate through the following equation (which represents the signal efficiency for requiring 0 
jets mis-tagged as b-jets):

\begin{equation}\label{eq:nttbar}
  \epsilon^{\textrm{NBtag} < 1} = 1 - \sum_{n=1} P(n) \cdot \sum_{m=1}^{n} C(n,m) \cdot f^{m} \cdot (1-f)^{n-m}
\end{equation}

where $P(n)$ is the probability to obtain $n$ additional jets (non-tau and non-lepton) in the event, $C(n,m)$ the 
combinatorial of $n$ $choose$ $m$, and $f$ the mis-tag rate. The probability to 
obtain at least one additional jet in the event is $\sim$ 10\%. Therefore, based on the above equation, the 
mis-tag rate and uncertainty, and the probability to obtain at least one additional jet we calculate a 
systematic effect of $\sim 5$\% on our signal due to the mis-tag rate. The b-tagging/mis-tagging systematics are considered 100\% correlated across MC based 
backgrounds with similar composition (e.g. W + Jets and DY + Jets where there are typically no real b-jets), but completely uncorrelated to backgrounds that have 
different composition (e.g. $t\bar{t}$ vs. DY + Jets).

  \item \textbf{Electron Energy Scale (EES):} We consider the effect on the signal acceptance efficiency of a 1\% (2.5\%) shift on the electron
  energy scale in the barrel (endcap) region. The resultant systematic uncertainty on signal and MC-based backgrounds is $< 1$\%.

  \item \textbf{Muon Momentum Scale (MMS):} We consider the effect on the signal acceptance efficiency of a 1\% momentum scale uncertainty on the
  muon momentum. The resultant systematic uncertainty on signal and MC-based backgrounds is $< 1$\%.

  \item \textbf{Tau Energy Scale (TES):} We consider the effect of the 5\% tau energy scale uncertainty measured by the tau 
POG on the signal acceptance. The energy component of the tau 4-momentum is scaled by a factor of $k=1.05$,  so that $p_{smeared} = k \cdot p_{default}$ and variables are recalculated using $p_{smeared}$. We find that by using $p_{smeared}$ calculated with 
a factor of $k=\pm 1.05$, the signal and MC-based backgrounds varies by up to $11$\%. 

  \item \textbf{Jet Energy Scale (JES):} We consider the effect of a 3-5\% jet energy scale uncertainty on the signal 
acceptance (depending on the $\eta$ and $p_{T}$ of the considered jet as prescribed by the $JetMET$ POG). The jet 
4-momentum is scaled by a factor of $k=1.05$, so that $p_{smeared} = k \cdot p_{default}$ and variables are recalculated 
using $p_{smeared}$. We find that by using $p_{smeared}$ calculated with
a factor of $k=\pm 1.05$, the signal and MC-based backgrounds varies by up to $12$\%. 
%  \item \textbf{MET:} The uncertainty on MET for our signal process is driven by the tau energy scale (TES), jet energy 
%scale (non-tau jets) (JES), light lepton energy/momentum scale (LES), and unclustered energy (UCE). 
%The systematic effect from MET due to TES, JES and LES is included in the JES, 
%TES, LES systematic uncertainties described above. We find that a 10\% uncertainty on the unclustered energy results 
%in at most a 0.5\% fluctuation on the signal acceptance.
%
%  \item \textbf{MET:} The uncertainty on MET for our signal process is driven by the jet energy 
%scale (non-tau jets) (JES), light lepton energy/momentum scale (LES), and unclustered energy (UCE).
%The systematic effect from MET due to TES, JES and LES is included in the JES,
%TES, LES systematic uncertainties described above. We find that a 10\% uncertainty on the unclustered energy results
%in at most a 0.5\% fluctuation on the signal acceptance.

%%  \item \textbf{PDF Systematics Uncertainty:} We consider the effect of the PDF uncertainties on the signal acceptance. 
%%Due to the lack of available PDF weights for the $Z^\prime$ signal samples, we slice up mass binned DY samples to mock up $Z^\prime$ samples in 
%%the following way:
%%\begin{table}[htbp!]	
%%  \begin{tabular}{| l | c | c | c | c | c | c |} 
%%  \hline\hline 
%%  $Z^\prime$ mass point (GeV) & 500 & 1000 & 1500 & 2000 & 2500 & 3000\\
%%  \hline
%%  DY mass slice (GeV) & 500 -- 550 & 1000 -- 1050 & 1500 -- 1550 & 2000 -- 2050 & 2450 -- 2550 & 2800 -- 3000\\
%%  \hline \hline
%%  \end{tabular}
%%\end{table}

%%Following the ``PDF4LHC recommendations for LHC Run
%%II"\cite{PDF4LHC15}, the PDF uncertainties are computed from the
%%68$\%$ confidence level with the PDF4LHC15$\_$mc sets. The PDF
%%uncertainties for our main backgrounds, $t\bar{t}$, W+Jets and DY, are
%%much smaller than their bin-by-bin statistical uncertainties and thus are
%%neglected. The PDF uncertainties on the signal acceptance range from
%%0.7$\%$ for $Z^\prime$ at 500 GeV up to 12$\%$ for $Z^\prime$ at 3 TeV.


  \item \textbf{Background Estimates:} The uncertainty on the data-driven background estimations are driven by the statistics in data in the various control
samples. There is also a mostly negligible contribution from the level of contamination
from other backgrounds in the control regions. In cases where MC based backgrounds must be subtracted off, the uncertainties in the MC backgrounds due to the above listed
systematic uncertainties are propagated throughout the subtraction and used to assign a systematic uncertainty on the background prediction.

\end{itemize}


\newcommand{\ch}{\tiny $hh$, $\mu h$, $eh$, $e\mu$}

\begin{table}[htbp!]\centering
 \caption{Summary of systematic uncertainties. Values are given in
   percent.  ``s'' indicates template variations (``shape''
   uncertainties).}
 \begin{tabular}{|l|c|c|c|c|c|c|} \hline \hline
   Source                 & QCD         & W           & DY          & \ttbar      & VV          & Signal      \\
                          & \ch         & \ch         & \ch         & \ch         & \ch         & \ch         \\
   \hline Luminosity            & --,--,--,-- & 5,--,5,5    & 5,5,5,5     & 5,5,5,5     & 5,5,5,5     & 5,5,5,5     \\
   \hline $\mu$ Trig      & --,--,--,-- & --,--,--,-- & --,1,--,--  & --,1,--,--  & --,1,--,--  & --,1,--,--  \\
   \hline $\mu$ ID        & --,--,--,-- & --,--,--,1  & --,1,--,1   & --,1,--,1   & --,1,--,1   & --,1,--,1   \\
   \hline e Trig          & --,--,--,-- & --,--,1,1   & --,--,1,1   & --,--,1,1   & --,--,1,1   & --,--,1,1   \\
   \hline e ID            & --,--,--,-- & --,--,1,1   & --,--,1,1   & --,--,1,1   & --,--,1,1   & --,--,1,1   \\
   \hline $\tau_{h}$ Trig & --,--,--,-- & 10,--,--,-- & 10,--,--,-- & 10,--,--,-- & 10,--,--,-- & 10,--,--,-- \\
   \hline $\tau_{h}$ ID   & --,--,--,-- & 30,--,6,--  & 12,6,6,--   & 12,6,6,--   & 12,6,6,--   & 12,6,6,--   \\
   \hline b ID            & --,--,s,s   & 10,--,s,s   & 3,3,s,s     & 10,12,s,s   & 3,3,s,s     & 3,3,s,s     \\
   %% \hline JES          & --,--,s,s   & 12,--,s,s   & 8,9,s,s     & 12,10,s,s   & 8,9,s,s     & 2,2,s,s     \\
   \hline JES             & --,--,s,s   & 12,--,s,s   & 8,s,s,s     & 12,s,s,s    & 8,s,s,s     & 2,2,s,s     \\
   \hline MMS             & --,--,--,-- & --,--,--,1  & --,1,--,1   & --,1,--,1   & --,1,--,1   & --,1,--,1   \\
   \hline EES             & --,--,--,-- & --,--,1,1   & --,--,1,1   & --,--,1,1   & --,--,1,1   & --,--,1,1   \\
   %% \hline TES          & --,--,s,s   & 11,--,s,s   & 11,7,s,s    & 11,9,s,s    & 8,8,s,s     & 3,3,s,s     \\
   \hline TES             & --,--,s,s   & 11,--,s,s   & 11,s,s,s    & 11,s,s,s    & 8,s,s,s     & 3,3,s,s     \\
   \hline Closure+Norm.   & 8,68,16,37  & 5,8,10,41   & 19,7,12,12  & 8,8,8,8     & 15,15,15,15 &             \\
   \hline \hline
 \end{tabular}
 \label{table:SystematicsTable}
\end{table}

