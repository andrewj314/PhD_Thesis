\section{Data and Monte Carlo Samples}\label{sec:samples}


The 13 TeV collision data collected by the CMS detector in year 2015 is used in this analysis. Table~\ref{table:datasamples} shows the collision datasets used. 
 The official CMS ``good run list", containing information on fill quality, is used to select ``good'' run ranges and lumi sections. The total integrated luminosity of the collision data samples is $2.11$ fb$^{-1}$.

The official Spring 2015 MC samples are used for all Standard Model processes. The leading order generators, {\sc Pythia8} and {\sc Madgraph}, were 
mainly used for Signal and Background MC production. The predicted background yields in simulation were determined using NLO or NNLO cross-sections, while the 
signal yields and distributions in all plots shown in this analysis were normalized using the leading order cross-sections shown in Table~\ref{tab:mc_samples}. 
Table~\ref{tab:mc_samples} shows the entire list of the MC samples used for this analysis. 
%In all cases, the NNLO cross sections are used for EWK processes. 
%The cross sections used are summarized in Table~\ref{table:mcxsections}.

\begin{table}[ht][H]	
  \caption{Collision Data Samples} 
  \centering{
  \begin{tabular}{| l | c |} 
  \hline\hline 
	Physics Sample & Official CMS Datasets \\ [0.5ex] \hline 

        \footnotesize Run 2015C SingleMu 05 Oct ReMiniAOD  & \footnotesize \it /SingleMuon/Run2015C\_25ns-05Oct2015-v1/MINIAOD \\
        \footnotesize Run 2015D SingleMu 05 Oct ReMiniAOD  & \footnotesize \it /SingleMuon/Run2015D-05Oct2015-v1/MINIAOD \\
        \footnotesize Run 2012D SingleMu PromptReco v4  & \footnotesize \it /SingleMuon/Run2015D-PromptReco-v4/MINIAOD \\

\hline
        \footnotesize Run 2015C SingleElectron 05 Oct ReMiniAOD  & \footnotesize \it /SingleMuon/Run2015C\_25ns-05Oct2015-v1/MINIAOD \\
        \footnotesize Run 2015D SingleElectron 05 Oct ReMiniAOD  & \footnotesize \it /SingleElectron/Run2015D-05Oct2015-v1/MINIAOD \\
        \footnotesize Run 2012D SingleElectron PromptReco v4  & \footnotesize \it /SingleElectron/Run2015D-PromptReco-v4/MINIAOD \\

\hline 
        \footnotesize Run 2015C Tau 05 Oct ReMiniAOD  & \footnotesize \it /Tau/Run2015C\_25ns-05Oct2015-v1/MINIAOD \\
        \footnotesize Run 2015D Tau 05 Oct ReMiniAOD  & \footnotesize \it /Tau/Run2015D-05Oct2015-v1/MINIAOD \\
        \footnotesize Run 2012D Tau PromptReco v4  & \footnotesize \it /Tau/Run2015D-PromptReco-v4/MINIAOD \\
%\hline
  \hline 
  \hline 
  \end{tabular}
  }
  \label{table:datasamples} % is used to refer this table in the text
\end{table}

\begin{table}[htbp!]
  \caption{MC Samples} 
  \centering{
  \resizebox{\textwidth}{!}{
  \begin{tabular}{| l | c | c |} 
  \hline\hline 
	Process & cross-section (pb) & Official CMS Datasets (MINIAODSIM) \\ [0.5ex] \hline

    \footnotesize $Z \to ll $   &  6025.2       & \footnotesize \it /DYJetsToLL\_M-50\_TuneCUETP8M1\_13TeV-madgraphMLM-pythia8/RunIISpring15MiniAODv2-74X\_mcRun2\_asymptotic\_v2-v1\\
    HT binned                   &  147.4*1.23   & \footnotesize \it /DYJetsToLL\_M-50\_HT-100to200\_TuneCUETP8M1\_13TeV-madgraphMLM-pythia8/RunIISpring15MiniAODv2-74X\_mcRun2\_asymptotic\_v2-v1/\\
    LO samples                  &  40.99*1.23   & \footnotesize \it /DYJetsToLL\_M-50\_HT-200to400\_TuneCUETP8M1\_13TeV-madgraphMLM-pythia8/RunIISpring15MiniAODv2-74X\_mcRun2\_asymptotic\_v2-v1\\
                                &  5.678*1.23   & \footnotesize \it /DYJetsToLL\_M-50\_HT-400to600\_TuneCUETP8M1\_13TeV-madgraphMLM-pythia8/RunIISpring15MiniAODv2-74X\_mcRun2\_asymptotic\_v2-v2\\
                                &  2.198*1.23   & \footnotesize \it /DYJetsToLL\_M-50\_HT-600toInf\_TuneCUETP8M1\_13TeV-madgraphMLM-pythia8/RunIISpring15MiniAODv2-74X\_mcRun2\_asymptotic\_v2-v1\\
    \hline 
    \footnotesize $Z \to ll $   &  6025.2       & \footnotesize \it /DYJetsToLL\_M-50\_TuneCUETP8M1\_13TeV-amcatnloFXFX-pythia8/RunIISpring15MiniAODv2-74X\_mcRun2\_asymptotic\_v2-v1\\
    mass binned                 &  7.67*0.987   & \footnotesize \it /DYJetsToLL\_M-200to400\_TuneCUETP8M1\_13TeV-amcatnloFXFX-pythia8/RunIISpring15MiniAODv2-74X\_mcRun2\_asymptotic\_v2-v1\\
    NLO samples                 &  0.423*0.987  & \footnotesize \it /DYJetsToLL\_M-400to500\_TuneCUETP8M1\_13TeV-amcatnloFXFX-pythia8/RunIISpring15MiniAODv2-74X\_mcRun2\_asymptotic\_v2-v1\\
                                &  0.24*0.987   & \footnotesize \it /DYJetsToLL\_M-500to700\_TuneCUETP8M1\_13TeV-amcatnloFXFX-pythia8/RunIISpring15MiniAODv2-74X\_mcRun2\_asymptotic\_v2-v3\\
                                &  0.035*0.987  & \footnotesize \it /DYJetsToLL\_M-700to800\_TuneCUETP8M1\_13TeV-amcatnloFXFX-pythia8/RunIISpring15MiniAODv2-74X\_mcRun2\_asymptotic\_v2-v1\\
                                &  0.03*0.987   & \footnotesize \it /DYJetsToLL\_M-800to1000\_TuneCUETP8M1\_13TeV-amcatnloFXFX-pythia8/RunIISpring15MiniAODv2-74X\_mcRun2\_asymptotic\_v2-v1\\
                                &  0.016*0.987  & \footnotesize \it /DYJetsToLL\_M-1000to1500\_TuneCUETP8M1\_13TeV-amcatnloFXFX-pythia8/RunIISpring15MiniAODv2-Asympt25ns\_74X\_mcRun2\_asymptotic\_v2-v1\\
    \hline 
    \footnotesize $W + Jets $   &  61526.7      &\scriptsize \it  /WJetsToLNu\_TuneCUETP8M1\_13TeV-madgraphMLM-pythia8/RunIISpring15MiniAODv2-74X\_mcRun2\_asymptotic\_v2-v1\\
    HT binned                   &  1345*1.21    &\scriptsize \it  /WJetsToLNu\_HT-100To200\_TuneCUETP8M1\_13TeV-madgraphMLM-pythia8/RunIISpring15MiniAODv2-74X\_mcRun2\_asymptotic\_v2-v1\\
    LO samples                  &  359.7*1.21   &\scriptsize \it  /WJetsToLNu\_HT-200To400\_TuneCUETP8M1\_13TeV-madgraphMLM-pythia8/RunIISpring15MiniAODv2-74X\_mcRun2\_asymptotic\_v2-v1\\
                                &  48.91*1.21   &\scriptsize \it  /WJetsToLNu\_HT-400To600\_TuneCUETP8M1\_13TeV-madgraphMLM-pythia8/RunIISpring15MiniAODv2-74X\_mcRun2\_asymptotic\_v2-v1\\
                                &  18.77*1.21   &\scriptsize \it  /WJetsToLNu\_HT-600ToInf\_TuneCUETP8M1\_13TeV-madgraphMLM-pythia8/RunIISpring15MiniAODv2-74X\_mcRun2\_asymptotic\_v2-v1\\
    \hline 
    \footnotesize $t \overline{t}$  &   831.76      &\scriptsize \it  /TT\_TuneCUETP8M1\_13TeV-powheg-pythia8/RunIISpring15MiniAODv2-74X\_mcRun2\_asymptotic\_v2\_ext3-v1\\
    \footnotesize single Top samples    &  35.6             &\scriptsize \it  /ST\_tW\_antitop\_5f\_inclusiveDecays\_13TeV-powheg-pythia8\_TuneCUETP8M1/RunIISpring15MiniAODv2-74X\_mcRun2\_asymptotic\_v2-v1\\
                                &  35.6             &\scriptsize \it  /ST\_tW\_top\_5f\_inclusiveDecays\_13TeV-powheg-pythia8\_TuneCUETP8M1/RunIISpring15MiniAODv2-74X\_mcRun2\_asymptotic\_v2-v2\\
                                &  136.02*0.108*3   &\scriptsize \it  /ST\_t-channel\_top\_4f\_leptonDecays\_13TeV-powheg-pythia8\_TuneCUETP8M1/RunIISpring15MiniAODv2-74X\_mcRun2\_asymptotic\_v2-v1\\
                                &  80.95*0.108*3   &\scriptsize \it  /ST\_t-channel\_antitop\_4f\_leptonDecays\_13TeV-powheg-pythia8\_TuneCUETP8M1/RunIISpring15MiniAODv2-74X\_mcRun2\_asymptotic\_v2-v1\\
    \hline 
    \footnotesize $VV $              &  11.95    &\scriptsize \it   /VVTo2L2Nu\_13TeV\_amcatnloFXFX\_madspin\_pythia8/RunIISpring15MiniAODv2-74X\_mcRun2\_asymptotic\_v2-v1\\
    \footnotesize $ZZ \to 2l 2q $    &  3.22     &\scriptsize \it   /ZZTo2L2Q\_13TeV\_amcatnloFXFX\_madspin\_pythia8/RunIISpring15MiniAODv2-74X\_mcRun2\_asymptotic\_v2-v1\\
    \footnotesize $ZZ \to 4l $       &  1.212    &\scriptsize \it   /ZZTo4L\_13TeV-amcatnloFXFX-pythia8/RunIISpring15MiniAODv2-74X\_mcRun2\_asymptotic\_v2-v1\\
    \footnotesize $WW \to l \nu 2q$  &  49.997   &\scriptsize \it  /WWTo1L1Nu2Q\_13TeV\_amcatnloFXFX\_madspin\_pythia8/RunIISpring15MiniAODv2-74X\_mcRun2\_asymptotic\_v2-v1\\
    \footnotesize $WZ \to 2l 2q$     &  5.595    &\scriptsize \it  /WZTo2L2Q\_13TeV\_amcatnloFXFX\_madspin\_pythia8/RunIISpring15MiniAODv2-74X\_mcRun2\_asymptotic\_v2-v1\\
    \footnotesize $WZ + Jets$        &  5.26     &\scriptsize \it  /WZJets\_TuneCUETP8M1\_13TeV-amcatnloFXFX-pythia8/RunIISpring15MiniAODv2-74X\_mcRun2\_asymptotic\_v2-v1\\
    \footnotesize $WZ \to l 3\nu$    &  3.05     &\scriptsize \it  /WZTo1L3Nu\_13TeV\_amcatnloFXFX\_madspin\_pythia8/RunIISpring15MiniAODv2-74X\_mcRun2\_asymptotic\_v2-v1\\
    \footnotesize $WZ \to l \nu 2q$  &  10.71    &\scriptsize \it  /WZTo1L1Nu2Q\_13TeV\_amcatnloFXFX\_madspin\_pythia8/RunIISpring15MiniAODv2-74X\_mcRun2\_asymptotic\_v2-v1\\
    \hline 
    \footnotesize $QCD$ samples &  720648000    &\scriptsize \it /QCD\_Pt20toInf\_MuEnrichedPt15\_TuneCUETP8M1\_13TeV\_pythia8/RunIISpring15MiniAODv2-74X\_mcRun2\_asymptotic\_v2-v1 \\
    \hline 
    \footnotesize $Z^\prime (500)$    &  9.33     &\scriptsize \it /ZprimeToTauTau\_M\_500\_TuneCUETP8M1\_tauola\_13TeV\_pythia8/RunIISpring15MiniAODv2-74X\_mcRun2\_asymptotic\_v2-v1 \\
    \footnotesize $Z^\prime (1000)$   &  0.468    &\scriptsize \it /ZprimeToTauTau\_M\_1000\_TuneCUETP8M1\_tauola\_13TeV\_pythia8/RunIISpring15MiniAODv2-74X\_mcRun2\_asymptotic\_v2-v1 \\
    \footnotesize $Z^\prime (1500)$   &  0.0723   &\scriptsize \it /ZprimeToTauTau\_M\_1500\_TuneCUETP8M1\_tauola\_13TeV\_pythia8/RunIISpring15MiniAODv2-74X\_mcRun2\_asymptotic\_v2-v1 \\
    \footnotesize $Z^\prime (2000)$   &  0.0173   &\scriptsize \it /ZprimeToTauTau\_M\_2000\_TuneCUETP8M1\_tauola\_13TeV\_pythia8/RunIISpring15MiniAODv2-74X\_mcRun2\_asymptotic\_v2-v1 \\
    \footnotesize $Z^\prime (2500)$   &  0.00554  &\scriptsize \it /ZprimeToTauTau\_M\_2500\_TuneCUETP8M1\_tauola\_13TeV\_pythia8/RunIISpring15MiniAODv2-74X\_mcRun2\_asymptotic\_v2-v1 \\
    \footnotesize $Z^\prime (3000)$   &  0.00129  &\scriptsize \it /ZprimeToTauTau\_M\_3000\_TuneCUETP8M1\_tauola\_13TeV\_pythia8/RunIISpring15MiniAODv2-74X\_mcRun2\_asymptotic\_v2-v1 \\

  \hline 
  \hline 
  \end{tabular}
  }
  }
  \label{tab:mc_samples}
\end{table}

Because the MC simulated samples contain a pileup (PU) distribution that does not match that of data, 
the MC needs to be properly weighted to fit the PU distribution observed in data. The reweighting 
of MC events is performed by determining the probabilities to obtain $n$ interactions in data 
($P_{data}(n)$) and MC ($P_{MC}(n)$) and using the event weights

\begin{equation}
   w_{PU}(n) = \frac{P_{data}(n)}{P_{MC}(n)}
\label{eq:PUweight}
\end{equation}

\noindent to reweigh MC events based on the number of interactions. 
In section~\ref{ss:puvalidation}, our understanding of PU and the performance of the PU reweighing method is validated.

