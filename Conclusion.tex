\chapter{Summary, Conclusion, and What's Next for $Z^\prime\to\tau\tau$}

This thesis has presented a brief overview of the Standard Model of particle physics, followed by a description of the CMS experiment at the Large Hadron Collider, followed by an in-depth description of the work done by the author and his collaborators over two major multiyear, multichannel searches for new particles at CMS. Both analyses were framed with high mass neutral resonances, $Z^\prime$s, decaying to $\tau^+\tau^-$ pairs as the searched-for signal in question. Motivated by multiple beyond-Standard Model theories, namely those that add an additional U(1) symmetry to the SM, these searches focused on the $Z^\prime\to\tau\tau$ decay channel. 

The first search used 19.7 fb$^{-1}$ of data collected during the 2012 CMS run at the LHC, during which time the center-of-mass energy was $\sqrt{s} = 8$ TeV. With a combination of Monte Carlo-based and data driven background estimation techniques, the distribution of ditau visible mass + $\MET$ was studied and no excess was observed. Due to issues related to reconstruction of hadronic taus, limits were only able to be set in the $\emu$ channel. At 95\% confidence level, no ditau excess decaying to an electron and muon is observed below 1.3 TeV for the $Z^\prime_{SSM}$ model and below 810 GeV for the $E_6$-inspired model.

The second search used 2.2 fb$^{-1}$ of data collected during the 2015 CMS run at the LHC, during which time the center-of-mass energy was $\sqrt{s} = 13$ TeV. This analysis considered all four decay channels, with the bulk of the author's work focused on the $\ditauhad$ channel. Similar techniques for background estimation as used in the 8 TeV analysis were employed here, and once again the distribution of ditau visible mass + $\MET$ was studied. No excess was observed, and limits were set on each individual channel. The combined limit across all channels excludes a $Z^\prime_{SSM}$ boson decaying to pairs of taus below 2.1 TeV at 95\% confidence level, the highest limit observed for a $Z^\prime$ decaying to taus. This analysis has been documented in a CMS analysis note (AN), a CMS physics analysis summary (PAS), and is currently undergoing collaboration-wide review (CWR) for publication as a CMS paper.

In addition to the core analyses, the author has devoted a large portion of his research career towards the development of additional selection criteria dedicated towards improved identification of tau leptons. This work was motivated by a central hypothesis that the tau, while having quite a short lifetime on the order of $10^{-13}$ s, is sufficiently long-lived to make its decay products, electrons, muons, and hadronic jets, distinguishable from the background electrons, muons, and jets originating from the primary vertex. This difference would manifest itself in those charged tracks originating from tau decays having a larger impact parameter with respect to the primary vertex (and later the beam spot) than those from ``prompt" decays. This concept was first tested in the $\emu$ channel with the 2012 8 TeV data, first with the simple sum of electron and muon impact parameters and then with more intricate combinations taking the track uncertainties into account. The effectiveness of cutting on the distance-of-closest-approach between the tracks rather than the impact parameter was studied, and the conclusion was reached that the two approaches yielded the highest overall signal-to-noise ratio when used in concert, requiring events to pass one or the other in the selection process. The application of these ``lifetime cuts" led to a significant suppression of prompt backgrounds and subsequent improvement of the 8 TeV $\emu$ limit by nearly 100 GeV. This study was again applied to the $\ditauh$ channel in the 2015 13 TeV analysis, where the same cut definitions were modified to act on the leading charged tracks in hadronic tau candidates. Similar performance was found in terms of prompt background suppression, but the sacrifice of signal events in the high-mass region (the region most sensitive to the $Z^\prime\to\tau\tau$ search) was severe enough that the limit suffered. Despite this, the lifetime cuts still have significant potential for future analyses involving taus.

The LHC and CMS have each performed beautifully this year. As of this writing, the scientists and engineers running the accelerator and the detector have recorded more than 30 fb$^{-1}$ of collision data for the 2016 run. By the time the LHC switches from $pp$ collisions to heavy ion collisions in early November, the total integrated luminosity will be closer to 40 fb$^{-1}$\cite{CMSLumi}. To put this in perspective, the total data collected during the 2015 run was 3.81 fb$^{-1}$. With this much data, the hope is that sufficient statistics will be available in the space of high-mass taus that future searches will be able to afford the loss in signal events and reap the high rate of background suppression demonstrated by the lifetime cuts. To put it succinctly: given enough events, the lifetime cuts should be quite effective at pushing the limit higher.

In addition to the lifetime cuts, there are additional planned modifications to the $Z^\prime\to\tau\tau$ search as the analysis is repeated with the 2016 data and beyond. A new series of topological cuts involving $\MET$ have been proposed. Since the neutrinos in the signal come from the tau decays, it is expected that the $\MET$ will be colinear with one of the tau legs (depending on the balance of $p_T$). One such topological cut requires the cosine of the angle between the $\MET$ and the difference in $p_T$ between the tau legs to be less than -0.9 ($\MET$ and $\Delta p_T$ pointing in opposite directions). Another requires the cosine of the angle between the $\MET$ and one of the tau legs to be greater than 0.9 ($\MET$ pointing along one of the tau decays). Finally, one recently-proposed cut would require the transverse mass of one of the tau legs and the $\MET$ to be greater than 150 GeV. Each of these cuts is designed to suppress $t\bar{t}$, QCD, and W+Jets backgrounds, as these are all processes in which the $\MET$ is not necessarily colinear with the leptons or jets faking tau decays.

The current LHC schedule cites a plan to increase the $pp$ center-of-mass collision to $\sqrt{s} = 14$ TeV, the full design energy of the collider by 2018. By the end of 2017, 100 fb$^{-1}$ is expected to be delivered to both CMS and ATLAS by the LHC. The LHC will then undergo its second long shutdown from mid-2018 to 2019, during which time upgrades to the injector and cryogenics systems are planned. By the end of 2022, a full 300 fb$^{-1}$ of collision data is expected to have been delivered. The third long shut down, from 2022 to 2025, will feature a host of upgrades to the LHC, CMS, and ATLAS in preparation for ``Phase 2" of LHC operation, also known as ``HL-HLC" (High-Luminosity LHC).\cite{LHCSchedule} During this phase, the LHC is expected to operate at 5- to 7-times the nominal luminosity. To handle this significant increase in luminosity, the Phase 2 upgrade for CMS in particular will involve several upgrades. These include a new pixel and strip trackers with integrated triggering, new endcap calorimeters in both ECAL and HCAL, a new L1 trigger system with improved latency, and new high-$\eta$ muon stations.\cite{CMSUpgrade} In particular, the CMS upgrade includes four additional pixel stations expected to deliver much higher impact parameter resolution. This has tremendous potential for the lifetime study as it could offer much better discrimination between tracks originating from tau decays and those originating from prompt vertices. By 2035, the HL-LHC is expected to deliver up to 3000 fb$^{-1}$ of data.\cite{LHCSchedule} 

Beyond the LHC, there are several additional colliders planned for construction around the world. The International Linear Collider (ILC) is a 500 GeV linear e$^+$e$^-$ collider with a proposal to begin construction in 2019 in Japan. The ILC would serve as a ``Higgs factory" with the chief goal of studying in greater detail the mass, spin, couplings, and other properties of the Higgs boson. In direct competition with the ILC is the Compact Linear Collider (CLIC), which is also a linear e$^+$e$^-$ collider with a proposed collision energy of 3 TeV. Further in the future, proposed colliders include a circular muon collider, which would allow higher lepton collision energies since the heavier muons would lose less energy due to synchotron radiation than the lighter electrons, and Very Large Hadron Collider (VLHC), a ~100 TeV $pp$ collider which would open the door to the discovery of new particles with masses an order of magnitude heavier than currently possible with the LHC.

With higher collision energies, new accelerators on the horizon, and thousands of femtobarns of scheduled data-taking, there is a rich and diverse space of unexplored physics to investigate. All of this, together with an abundance of talented young researchers, points to a future in high energy physics that is indeed bright.