\chapter{Summary, Conclusion, and What's Next for CMS}

This thesis has presented a brief overview of the Standard Model of particle physics, followed by a description of the CMS experiment at the Large Hadron Collider, followed by an in-depth description of the work done by the author and his collaborators over two major multiyear, multichannel searches for new particles at CMS. Both analyses were framed with high mass neutral resonances, $Z^\prime$s, decaying to $\tau^+\tau^-$ pairs as the searched-for signal in question. Motivated by multiple beyond-Standard Model theories, namely those that add an additional U(1) symmetry to the SM, these searches focused on the $Z^\prime\to\tau\tau$ decay channel. 

The first search used 19.7 $fb^{-1}$ of data collected during the 2012 CMS run at the LHC, during which time the center-of-mass energy was $\sqrt{s} = 8$ TeV. With a combination of Monte Carlo-based and data driven background estimation techniques, the distribution of ditau visible mass + $\MET$ was studied and no excess was observed. Due to issues related to reconstruction of hadronic taus, limits were only able to be set in the $\emu$ channel. At 95\% confidence level, no ditau excess decaying to an electron and muon is observed below 1.3 TeV for the $Z^\prime_{SSM}$ model and below 810 GeV for the $E_6$-inspired model.

The second search used 2.2 $fb^{-1}$ of data collected during the 2015 CMS run at the LHC, during which time the center-of-mass energy was $\sqrt{s} = 13$ TeV. This analysis considered all four decay channels, with the bulk of the author's work focused on the $\ditauhad$ channel. Similar techniques for background estimation as used in the 8 TeV analysis were employed here, and once again the distribution of ditau visible mass + $\MET$ was studied. No excess was observed, and limits were set on each individual channel. The combined limit across all channels excludes a $Z^\prime_{SSM}$ boson decaying to pairs of taus below 2.0 TeV at 95\% confidence level, the highest limit observed for a $Z^\prime$ decaying to taus.

In addition to the core analyses, the author has devoted a large portion of his research career towards the development of additional selection criteria dedicated towards improved identification of tau leptons.