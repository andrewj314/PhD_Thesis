\subsection{Post-unblinding: further checks of the backgrounds}
\label{sec:et_bkg_validation}

%% \begin{figure}\centering
%%   \includegraphics[width=0.45\textwidth]{figures/et-em/eTauStudies/excessCloseUp}
%%   \caption{\label{fig:closeUp} The distributions of
%%     reconstructed parent mass between 300 to 600 GeV, \teth channel:
%%     \meffetau.}
%% \end{figure}

%% The background composition for \meffetau in region [300, 600] is shown
%% in Figure~\ref{fig:closeUp}. As shown, the dominating background in
%% this region is W+Jets, consisting of around 55\% of the total
%% background, followed by QCD accounting for 19\% and $t\bar{t}$
%% accounting for 16\% of the total background.

\subsubsection{Checks of MC-based W+Jets}
Based on $m_T$ distribution shown in
Figure~\ref{fig:etau_sm_template_and_mt}, we require $60~\text{GeV} <
m_T < 120~\text{GeV}$ to compare data v.s estimated background in a
W+Jets rich region. Figure~\ref{fig:60_mT_120} shows that the W+Jets
shape estimated from the relaxed \tauh isolation region provides a
smoother template and agrees better with observations.
\begin{figure}\centering
  \includegraphics[width=0.45\textwidth]{figures/et-em/eTauStudies/WJets_non_Iso_60_mt_120}
  \includegraphics[width=0.45\textwidth]{figures/et-em/eTauStudies/WJets_iso_60_mt_120}
  \caption{\label{fig:60_mT_120} \teth channel, in the signal region,
    requiring also $60~\text{GeV} < m_T < 120~\text{GeV}$.  Left: The
    distributions of \meffetau with W+Jets shape estimated from
    relaxed \tauh isolation region. Right: The distributions of
    \meffetau with W+Jets shape estimated from tight \tauh isolation
    region.}
\end{figure}

%% A direct comparison of the W+Jets tight and sideband shapes, in three
%% slices of $m_T$, is shown in Figure~\ref{fig:et_WJets_mT}. When
%% $60~\gev < m_T < 120~\gev$, there appears to be a significant upward
%% fluctuation of W+Jets in the tight \tauh isolation region for
%% \meffetau regions of [300, 600].  In the other two slices of $m_T$,
%% the isolated and anti-isolated shapes look compatible within
%% statistics.

%% \begin{figure}\centering
%%   \includegraphics[width=0.32\textwidth]{figures/et-em/eTauStudies/WJets_iso_vs_non-Iso_mt_60}
%%   \includegraphics[width=0.32\textwidth]{figures/et-em/eTauStudies/WJets_iso_vs_non-Iso_60_mt_120}
%%   \includegraphics[width=0.32\textwidth]{figures/et-em/eTauStudies/WJets_iso_vs_non-Iso_120_mt}

%%   \caption{\label{fig:et_WJets_mT} \teth channel: comparison of the
%%     simulated W+jets distributions of \meffetau in the signal region,
%%     using tight \tauh isolation (red) or the \tauh isolation sideband
%%     from ``tight'' to 5\gev (blue).  Left: $m_T < 60\gev$. Center:
%%     $60\gev < m_T < 120\gev$.  Right: $120\gev < m_T$.}
%% \end{figure}

%% For $m_T > 120~\text{GeV}$, W+Jets shape comparisons and distributions
%% of \meffetau are shown in Figure~\ref{fig:120_mT}.  Due to low
%% statistics, it is difficult to conclude if there are any excess in
%% \meffetau regions [300, 600]. Overall, W+Jets shape estimated from the
%% relaxed \tauh isolation region agrees better with observations.

%% \begin{figure}\centering
%%   \includegraphics[width=0.45\textwidth]{figures/et-em/eTauStudies/WJets_non_Iso_120_mt}
%%   \includegraphics[width=0.45\textwidth]{figures/et-em/eTauStudies/WJets_iso_120_mt}
%%   \caption{\label{fig:120_mT} \teth channel, in the signal region,
%%     requiring also $m_T > 120~\gev$: Left: the distributions of
%%     \meffetau with W+Jets shape estimated from relaxed \tauh isolation
%%     region. Right: the distributions of \meffetau with W+Jets shape
%%     estimated from tight \tauh isolation region.}
%% \end{figure}

%% In the signal rich region, with $m_T < 60~\text{GeV}$,
%% Figure~\ref{fig:mT_60} shows that the excess seen in the high
%% \meffetau region cannot be explained by W+Jets MC shape variations
%% between relaxed and tight \tauh isolation regions.
%% \begin{figure}\centering
%%   \includegraphics[width=0.45\textwidth]{figures/et-em/eTauStudies/WJets_non_Iso_mt_60}
%%   \includegraphics[width=0.45\textwidth]{figures/et-em/eTauStudies/WJets_iso_mt_60}
%%   \caption{\label{fig:mT_60} \teth channel, in the signal region
%%     requiring also $m_T < 60~\gev$.  Left: The distributions of
%%     \meffetau with W+Jets shape estimated from relaxed \tauh isolation
%%     region. Right: The distributions of reconstructed parent mass with
%%     W+Jets shape estimated from tight \tauh isolation region. \teth
%%     channel}
%% \end{figure}

For the \teth channel, 2-prong \tauh's were rejected due low signal
presence and high W+Jets and QCD contamination.  Hence, 2-prong
\tauh's provide a testing group for the background estimation
methods. As shown in Figure~\ref{fig:et_2prong} there is a good
agreement between observations and estimated background.

\begin{figure}\centering
  \includegraphics[width=0.45\textwidth]{figures/et-em/eTauStudies/2prong}
  \caption{\label{fig:et_2prong} In the signal region, but requiring
    2-prong \tauh's: the distribution of reconstructed parent mass,
    \meffetau, with W+Jets shape estimated from relaxed \tauh
    isolation region.}
\end{figure}

\subsubsection{Data-driven W+jets checks}
Similar to the \tmth channel, by requesting e and \tauh to have opposite charge, 
we have the following four regions:
\begin{itemize}
  \item A (Signal) Region: pass "$\zeta$" and "$\cos\Delta\phi$" and \tauh pass "Tight" isolation requirement.
  \item B Region: fail "$\zeta$" or "$\cos\Delta\phi$" and \tauh pass "Tight" isolation requirement.
  \item C Region: pass "$\zeta$" and "$\cos\Delta\phi$" and \tauh pass anti-isolation requirement.
  \item D Region: fail "$\zeta$" or "$\cos\Delta\phi$" and \tauh pass anti-isolation requirement.
\end{itemize}
where the W+jets in the signal region is estimated from region C 
by subtracting all other background from data and multiplying a 
scale factor estimated among region B and D.
\begin{equation}\label{eq:et_data_WJets}
N^{\text{signal}}_{\text{W+jets}} = N^{\text{C}}_{\text{data - other bkg}} \times \frac{N^{\text{B}}_{\text{data - other bkgs}}}{N^{\text{D}}_{\text{data - other bkg}}}
\end{equation}

QCD in each region is estimated from data by inverting the charge
requirement.  Thus, we have four more regions A', B', C' and D' with
similar requirements as ABCD but requesting e and \tauh to have the
same charge. Hence, QCD in B, C and D regions are estimated in the
following way:
\begin{itemize}
  \item QCD in B Region: shape taken from data - all MC bkg (including W+jets MC) in D' region. 
Yield normalized to data - all MC bkg (including W+jets MC) in D region with a "SStoOS" scale 
factor of 1.07 $\pm$ 0.05 estimated from the \thth channel.
  \item QCD in C Region: shape taken from data - all MC bkg (including W+jets MC) in C' region. 
Yield normalized to data - all MC bkg (including W+jets MC) in C region
  \item QCD in D Region: shape taken from data - all MC bkg (including W+jets MC) in D' region. 
Yield normalized to data - all MC bkg (including W+jets MC) in D region.
\end{itemize}

The right panel of Figure~\ref{fig:dataDrivenWJets} shows the observation and background 
estimation comparisons with the data-driven W+jets and QCD estimations mentioned above. Compared 
with the MC W+jets estimation on the left panel, we see negligible difference between the 
two.

\begin{figure}\centering
  \includegraphics[width=0.45\textwidth]{figures/et-em/eTauStudies/et_puweight_OS_signalRegion_MCWJets_1and3prong}
  \includegraphics[width=0.45\textwidth]{figures/et-em/eTauStudies/et_puweight_OS_signalRegion_dataDrivenWJets_1and3prong}
  \caption{\label{fig:dataDrivenWJets} Left: W+jets shape and yield from simulation
    Right: Data-driven W+jets estimation \teth
    channel: \meffetau}
\end{figure}
