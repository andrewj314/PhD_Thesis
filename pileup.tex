\subsection{Corrections for Pile-Up}\label{ss:puvalidation}

Quantities such as \MET ~and isolation, where energy depositions are
summed up over some range of the detector, can suffer large
inefficiencies or systematic effects due to particles from pile-up
interactions. Therefore, a simple and robust method must be employed
to subtract off the contribution from secondary interactions.  In the
case of \MET, this is done by using the FastJet corrections to
determine the density of PU on an event by event basis. For isolation, the 
recommended $\delta\beta$ corrections are applied. The
recommended PU corrections have been included for all objects in all
the channels.

%For taus, $\Delta\beta$ corrections are defined as:

%\begin{equation}
%   I = \sum_{i} P_{T}^{i, charged} + max(E_{T}^{i, gamma} + E_{T}^{i, neutral} - 0.5 \times E_{T}^{PU} , 0.0)
%\end{equation}

%where $E_{T}^{PU}$ is the charged particle $p_{T}$ sum from PU vertices. The $\Delta\beta$ correction uses 
%the fact that the contribution to isolation from neutral particle deposits can be determined by using the 
%percentage of particle flow charge hadrons considered for isolation that arise from PU. The recommended 
%PU corrections have been included for all objetcs in all the channels.

We apply event-by-event the ``official'' pile-up weights
($\sigma=69$~mb).  Figure~\ref{fig:nvtx} shows the distributions of
the number of reconstructed vertices, before and after applying the
pile-up weights.

\begin{figure}
  \centering
  \includegraphics[width=0.4\textwidth]{figures/et-em/PUReweighting/13TeV_em_OS}
  \includegraphics[width=0.4\textwidth]{figures/et-em/PUReweighting/13TeV_em_puweight_OS} \\
  \includegraphics[width=0.4\textwidth]{figures/et-em/PUReweighting/13TeV_et_OS_1and3prong}
  \includegraphics[width=0.4\textwidth]{figures/et-em/PUReweighting/13TeV_et_puweight_OS_1and3prong} \\
  \caption{\label{fig:nvtx} Distributions of the number of
    reconstructed primary vertices.  Top: \emu ~channel.  Bottom: \etau ~channel.  Left: before reweighting.  Right: after reweighting
    (``official'' 69~mb).  The ratios after reweighting become flatest
    at an ``unofficial'' value of 71~mb.  The apparent substantial
    differences in the QCD ``integrals'' are artifacts of the plotting
    procedure, in which bins with negative contributions are zero-ed.}
\end{figure}
