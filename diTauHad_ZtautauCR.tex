\subsection{\texorpdfstring{Background Estimation for $Z (\rightarrow \tau\tau)$ + Jets}{Background Estimation for Z (to tau tau) + Jets}}

We do not employ a complete data-driven measurement of the $Z\to\tau\tau$ + Jets contribution to the signal region. Instead, we take an approach based on
both simulation and data. The efficiency for the requirement of at least two high quality $\tau_{h}$s is expected to be well modeled by simulation. 
Therefore, the estimate of the $Z\to\tau\tau$ + Jets contribution is determined by obtaining a control sample used to validate the 
correct modeling of the requirement of at least two high quality $\tau_{h}$s. 
Since the DY + Jets background in this channel is 
$< 10$\% of the total background in the signal region, the above approach is sufficient. 

As discussed above, the typical probability of misidentifying a QCD jet as a $\tau_{h}$ is at least
an order of magnitude higher than that for a QCD jet to be misidentified as a light lepton. As a
result the QCD multijet background in the $\tau_{h}\tau_{h}$ channel is substantially higher
than in lepton plus tau or dilepton channels. One should note that the presence
of large multijet background mainly complicates the definition of suitable control regions for
validating the agreement between collision data and simulation for other backgrounds.
For this purpose, the events are selected using the ``pre-selection" cuts, and additionally requiring $\tau_{h}\tau_{h}$ pairs with
invariant mass less than 100 GeV in order to obtain a semi-clean sample of Z$\to\tau\tau$ events. 
Figure~\ref{fig:ZtautauControlDiHad} shows the $m(\tau_{h}\tau_{h})$ distribution for this validation sample where the QCD contribution has been determined using
the method discussed above. One can see that the rate and shape between data and MC is consistent. The measured Z$\to\tau\tau$ data-to-MC scale factor is {\bf 
$SF_{\textrm{preselection}} = 0.97 \pm 0.19$}.

\begin{figure}[tbh!]
  \centering
    \includegraphics[width=0.75\textwidth]{figures/HeavyNu_13TeV_ZtautauCR.png}
  \caption{ $m(\tau_{h}\tau_{h})$ in the Z$\to\tau\tau$ control region obtained using the ``pre-selection" cuts, and additionally requiring 
$\tau_{h}\tau_{h}$ pairs with invariant mass less than 100 GeV.}
  \label{fig:ZtautauControlDiHad}
\end{figure}
